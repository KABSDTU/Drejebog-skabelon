\section{Førstehjælp}
\subsection{Førstehjælpskassen:}
Vi har en førstehjælpskasse med fra badekaret.


\subsection{Almindelig førstehjælp på rustur:}
Følgende ting kan vi komme ud for på rusturen og skal kunne håndtere:


\textbf{Fremmedlegemer i luftvejene:} Ved delvist blokeret (patienten kan godt sige noget) opfordres til at hoste og evt. slag med flad hånd mellem skulderbladende, samtidig med hoste. Ved total blokering anbefales det, at der veksles mellem at give 5 slag mellem skulderbladende og 5 hårde stød i bughulen op mod mellem gulvet, Heimlich (armene rundt om patient der bøjer sig forover). Husk at man skal på hospitalet bagefter, hvis der bliver udført Heimlich på en.


\textbf{Hjernerystelse:} Kommer til udtryk ved en eller flere af følgende symptomer: Hovedpine, Kvalme, Opkast, Svimmelhed, Synsforstyrrelse, Træthed, og tab af korttidshukommelse.


\textbf{Førstehjælpen består af:} Læg personen for at slappe af/sove og hold tæt observation i mindst 24 time, f.eks. ved at vække personen en gang i timen. Dette er for at sikre, at tilstanden ikke forværres. Bliver personen tiltagende dårlig, skal der søges skadestue eller læge hurtigst muligt.


\textbf{Forbrænding:} Mindre forbrændinger skal straks køles ned med, så koldt vand, som patienten kan holde ud, indtil det ikke gør ondt længere. Sværere forbrændinger (2. grads kan ses ved at der er vabler/sår og ved 3. grads vil huden være læderagtigt og blødende) skylles med tempereret vand (ca. 18-19 grader), der alarmeres og skylning forsætter under transport til videre behandling. 


\textbf{Hedeslag/solstik:} Symptomer hovedpine, svimmelhed og træthed. Huden vil være varm og lyserød. Patienten vil svede og være konfus. Førstehjælpen er at få patienten i skygge og fjern/løsne tøj. Læg koldt omslag ved pande, nakke, håndled, lyske og ankler, giv noget koldt at drikke. 


\textbf{Knoglebrud:} Ved ben- eller bækkenbrud tilkald ambulance og støt bruddet i findestillingen. Ved mindre brud kan man køre selv og holde bruddet i ro, evt med trekant tørklæde.


\textbf{Forstuvning:} Opstår ved for stor og forkert belastning af et led. Symptomerne kan afhjælpes ved at benytte sig af R.I.C.E-princippet:
\begin{itemize}
    \item R: Rest. Hold det beskadigede område i ro i det første døgn
    \item I: Ice. Afkøl med is(køleposer kan bruges) eller koldt vand. Køl i 30 min. og hold derefteren times pause. Fortsæt dette til hævelsen er aftaget.
    \item C: Compression. Sørg for at lægge pres på det beskadigede sted øjeblikkeligt, hold det i ca. 10-20 min og anlæg derefter et støttebind.
    \item E: Elevation: Løft det beskadigede område over hjertehøjde.
\end{itemize}

\textbf{Snitsår:} Rens med renseservietter og sæt plaster på, brug evt. lille kompresforbinding. Løft over hjertehøjde for at stoppe blødningen.


\textbf{Næseblod:} Pres 2 fingre på næsen og bøj hoved let forover. Læg evt. en kold klud eller is over næsen, for at få blodkar til at trække sig sammen. Hvis blødningen ikke er stoppet efter en halv time så søg læge eller skadestue.


\textbf{Flåt:} Fjern flåten med flåt-tang. Mas aldrig på kroppen af flåten. Kan den ikke fjernes, søg læge. Tegn cirkel omkring flåten før den fjernes, så kan det undersøges der spreder sig et mærke.\\
\newline
\textbf{Bistik:} Hvis man bliver stukket i munden eller hals skal der søges læge eller skadestue, er dette langt væk, tilkald ambulance. Normale bistik kan afhjælpes med en sukkerknald, der trækker giften ud, husk at se om brodden er kommet ud. 


\textbf{Astma:} Ved anfald, støt personen i at sidde eller stå. De skal ikke ligge ned, da dette kan forværre åndenøden. Armene over hovedet kan også hjælpe. Sørg for frisk luft og giv psykisk førstehjælp. Evt. hjælp med personens egen medicin eller astmaspray 


\textbf{Diabetes:} Skyldes nedsat eller ingen produktion af hormonet insulin der transporterer sukker fra blod ind i cellerne. Anfald kan skyldes lavt eller højt blodsukker. Ved lavt blodsukker gives noget sødt at spise eller drikke. Ellers trinvis førstehjælp. Er man i tvivl om det er højt eller lavt blodsukker, så giv noget sødt alligevel, de det ikke gør en stor forskel, hvis blodsukkeret allerede er højt. Man må ikke selv give insulin til sukkersygepatienter. Hvis de er ved bevidsthed kan man hjælpe, de har som regel selv helt styr på det.


\textbf{Epilepsi:} Få personen til at ligge ned og løsne stramt tøj om halsen.  Beskyt hovedet med stød når krampen ophører lægges personen i stabilt sideleje for at skabe frie luftveje. De vil oftest være meget trætte. Forsøg ikke at stikke noget mellem tænderne, men forhold dig i ro. 


\textbf{Allergi:} Allergi skyldes, at kroppen reagerer på stoffer som fødevarer, kemikalier, medikamenter eller pollen, som for de fleste er uskadelige. De mest almindelige symptomer er åndedrætsbesvær, hududslæt, mavesmerter eller opkastninger og diarré. Nogle kan gå i anafylaktisk shock (livstruende reaktioner). 
Ved milde symptomer hjælpes med, at den ramte får taget sin medicin. Håndkøbsmedicin, som antihistaminer, bør tages, hvis muligt.

Ved alvorlige allergisk reaktioner skal der spørges om personen har medicin mod allergisk shock og hjælpes med, at personen får taget dette og ringes 112. Ved bevidstløshed lægges i stabilt sideleje. Fødderne skal hæves ved muligt. Ved manglende vejrtrækning påbegyndes genoplivning.


\textbf{Genoplivning:} Hvis en person ikke trækker vejret og eller hjertet stopper, skal genoplivning straks påbegyndes. Der skal skabes frie luftveje, kunstigt åndedræt og hjertemassage gives. Der benyttes HLR (hjerte/lungeredning), en hjertestarter lokaliseres og der ringes efter 112.
Først tjekkes bevidsthed, luftveje og åndedræt.


Der tjekkes for om det er muligt at kommer i kontakt med personen, om personen har noget galt i halsen (hvilket fjernes inden HLR påbegyndes, se Fremmedlegemer i luftvejene), til sidst tjekkes åndedrættet ved at se om brystet bevæger sig og lytte efter vejrtrækning. 


Brug ikke tid på at tjekke puls, da dette er usikkert!


Når HLR påbegyndes: Først råbes der efter hjælp og der sendes en afsted efter hjertestarter, personen lægges på ryggen hvis muligt. Derefter skabes frie luftveje ved at bøje hovedet godt bagover til naturlig modstand mødes og løfte hagen frem.


Herefter påbegyndes hjertemassage. Man lægger sig på knæ helt ind til personens arm. Hænderne anbringes ovenpå hinanden på brystkassen. Pres ca. 5 cm lodret ned og slip derefter igen. Hold albuerne strakte for at gøre dette lettere. Tryk 30 gange, med ca. 2 tryk i sekundet (100-120 tryk i minuttet). Tæl højt for at holde styr på det.


Efter 30 tryk gives mund-til-mund. Hovedet bevæges godt bagover til naturlig modstand mødes. Klem personens næsebor med to fingre. Løft hagen godt op med den anden hånd. Træk vejret dybt og anbring læber rundt om patientens mund. Blæs indtil brystkassen rejser sig. Dette gentages 2 gange og skal helst tage under 10 sekunder.


Hjertemassage og mund-til-mund gentages indtil ambulance ankommer. Skift gerne med en anden efter nogle minutter for at holde energiniveauet. 


Når der ringes 112: Prøv at optræde behersket og effektivt under telefonsamtalen.
Opgiv hvem der ringer, og hvor du ringer fra (adresse)


Hvad er der sket?


Nøjagtig angivelse af sted, aftal eventuelt mødested.


Få meldingen bekræftet af den du taler med.


\textbf{Hvor fa'en er den nærmeste Hjertestarter???}\\
Den nærmeste hjertestarter er placeret ved \placeholder



