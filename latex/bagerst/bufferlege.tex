\documentclass[../../../main.tex]{subfiles}
\begin{document}

\section{Bufferlege}
\begin{multicols}{2}
\subsection*{Frostbox}
    Den eneste regel er, at man ikke snakker om reglerne til Frostbox.

\subsection*{Meyer}
    Man slår med to terninger, uden at vise resultatet til de andre. Man siger den højeste kombination af tallene og giver bærget videre. Rækkefølgen går som følgende: almindelige tal, par, lille meyer (13), meyer (12). 32 er fællesskål, og man får lov til at prøve igen. Man må gerne lyve, men løbe risikoen at modstanderen lurer ens bluff.

\subsection*{Hygge meyer}
    Dette er et gentlemanspil, hvor målet er at slå lavest med terninger (samme pointsystem som i meyer). Vinder man får man en kop fyldt med læske, som står i midten af bordet. Da det er et gentlemanspil kan ens modspillere tillade en at kaste terningerne flere gange, hvis de mener at man kan gøre det bedre. Dog højst op til 3 gange (for første deltager). Ens modspillere har lige så mange forsøg til at kaste, som første spiller fik. Terningerne går rundt til alle spillere, og først der bestemmer man vinderen. Slår man lille meyer bunder man drikken i midten, fylder den op igen og slår om. Slår man meyer giver man bunderen til en anden, som efterfølgende fylder den op. Personen der slog meyer slår nu igen. Der bestemmes stadigvæk en vinder af runden på normal vis.
    
    Når man sender terningen videre siger man sit tal, opløftet i antallet af slag første deltager brugte. (f.eks. 41 i anden). Terningerne er ikke tildækkede når man slår.

\subsection*{Ninjawarrior}
    Alle stiller sig i en cirkel. Man får lov til at slå ud efter de andres hænder med en bevægelse. Disse må undvige. Hvis man rammer mister de den pågældne hånd. Man har altså 2 liv. Turen går videre i positiv omløbsretning indtil der kun er en spiller tilbage.

\subsection*{Petite}
    Alle stiller sig op i en cirkel og holder hinanden i skuldrene. Man synger "petite, petite, petite petite petite petite" og tager skridt i tagt. Derefter stopper man, tager fat i næste persons skuldre i stedet og gentaget sangen. Dette fortsættes indtil alle kan sætte sig på lårene på personen bag sig.

\subsection*{Over broen}
    En deltager vælger et tema, f.eks. dyr, ting med is, Sverige etc. De andre skal så gætte temaet ved at spørge: "Må jeg tage X med over broen?", hvortil den med temaet svarer "Ja" - hvis det passer med temaet, eller "nej" hvis det ikke gør.
    Alle får et gæt, man kører rundt indtil man gætter eller giver op.
    Deltagerne diskutere åbent hvad man tror temaet er.
    
    \begin{itemize}
        \item Øjenkontakt
        \item Siger "øhm" først
        \item Noget med en motor
        \item Ting der kan komme i en drink
        \item Ting der ex. er grønne (eller en anden farve)
        \item Dyr der har 4 ben
        \item Ting der let kan brænde
        \item Ting der er cultural appropriation (f.eks sombrero, kimono osv.)
    \end{itemize}

\subsection*{Monsterleg}
    Vektor danner hold.
    Vektor råber et antal kropsdele. F.eks. 3 fødder, 5 hænder, 1 fabolous ass.
    Nu skal russerne sammenarbejde for kun at have det nævnte antal kropsdele på jorden.
    Vektor skal tjekke, at det stemmer overens før holdets “Monster” bliver godkendt.

\subsection*{Formation}
    Vektor råber en genstand. f.eks. HUS!
    Det gælder nu for russerne om at lave et hus.
    Det er lettere hvis de laver formationen liggende.
    Den der gør det hurtigst vinder.
    
\subsubsection*{Ubåd/Torpedo}
Folk deles ud på små hold. Disse hold stille sig på række med hænderne på skuldrene af hinanden, et godt stykke fra hinanden. Alle på holdet, på nær den bagerste, har lukkede øjne. Den bagerste skal forsøge at styre sit hold i retningen af de andre "ubåde", og ramme dem. Den bagerste klemmer i enten højre eller venstre skuldrer, og klemmet bliver sendt videre igennem rækken, for at rette sin egen ubåd. Torpedoen bliver skudt af sted, når den bagerste klemmer begge skuldre samtidigt. Hvis man rammer er holdet ude og sådan bliver man ved indtil der ikke er flere ubåde. 

\subsubsection*{Evolutionsleg}
Æg, Hammerhaj, Kat, Albatros.\\
Man starter ud med at hoppe rundt, på hug og siger "æg, æg, æg" og når man støder ind på ét andet æg skal man spille "sten, saks, papir", til 1, om hvem der bliver til en hammerhaj. Når man er blevet haj svømmer mand rundt foroverbøjet, med en arm på tværs foran hovedet, imens man siger "Hammer, Hammer, Hammer". Sten, saks, papir endnu engang når man møder en haj. \\

Så bliver man til kat og siger "Miauw". \\

Derefter bliver man til Albatros og flyver nu rundt med sine vinger og siger "Squark!" og har vundet! Tillykke

\subsubsection*{Mingle leg}
Går rundt og siger: Mingle, Mingle, Mingle også bliver der kaldt et tal: 5 også skal russerne gå i grupper af 5. De som ikke får dannet en gruppe er ude.

\subsubsection*{Alle mine kyllinger kom hjem}
Man stiller sig op på en række i den ene ende af banen. Midt på banen står ræven. Personen, der styrer legen, kalder, at alle, der opfylder et bestemt krav (fx har røde sko på), skal komme hjem. Dem, der opfylder kravet, skal løbe over på den anden side af banen. Ræven skal så forsøge at fange dem, der løber. Alle, der bliver fanget, skal stå på midten af banen og fange i de kommende rundter. Den sidste, der fanges, har vundet.

\subsubsection*{Kluddermor}
Én vælges til at være kluddermor. Personen går lidt væk og lukker øjnene. De andre holder hinanden i hænderne og vikler sig sammen. Kluddermor kaldes ind for at kludre dem ud.

\subsubsection*{Ståtrold}
En eller et par stykker vælges til at være dem, der skal fange de andre. Når man bliver fanget, skal man stille sig med spredte ben og armene ud til siden (som en sprellemand). Man bliver reddet ved, at en anden, fri person, kravler gennem benene på en. Personen, der kravler gennem en anden persons ben, skal sige personens navn. Hvis f.eks. Myggen kravler mellem Smølfs ben, skal Myggen sige “Smølf”.

\subsubsection*{Hyggestemning}
Ligge på jorden, med hovederne på hinandens maver. Primært til hyggestemning, kan ikke bruges som underholdning.

\subsubsection*{Frostbox}
Man forklarer ikke reglerne i frostbox


\subsubsection*{Massagestemning}
Dan en rundkreds og giv hinanden massage.


\subsubsection*{Molekyleleg}
Man skal være mange om denne leg. Antallet er ligegyldigt, bare man er mange. Der vælges en råber og denne råber atomer, alle deltager går nu rundt imellem hinanden. Råberen råber molekyler, og man finder sammen 2 og 2, den der er tilbage går ud. Råberen råber 5-molekyler. Alle går sammen 5 og 5 i en kæde med hinanden i hånden. Dem er ikke kan slutte sig til en hel kæde er ude. Råberen råber 12-molekyler. Legen fortsætter til enten kæden er så stor at ikke alle kan deltage eller at alle er gået ud. 
    
\subsubsection*{Studiestartsdans}{Hvis folk mangler at røre sig}
    Pongdance, Birdmachine, Zidansen, Left outside alone, Tamagotchi osv.
    
\subsubsection*{Ølkroket}{Hvis folk mangler at drikke}
    Drik mofufus \\



\subsubsection*{Ølhat}
\begin{itemize}
    \item Frisbee
    \item Ca. en milliard øl
\end{itemize}

Man stiller sig i en så god cirkel som man nu engang kan afhængig af hvor mange man lige er med til det (det kan sagtens spilles 1v1 men det er ikke ligefrem en overbevisende cirkel man så laver). 

Alle deltagere balancerer en øl på hovedet som nu er deres hat. En deltager tager frisbeen og kaster den efter en anden deltager. Det er tilladt at tage sin hat af når man kaster. Efter kastet må deltageren der er kastet efter tage frisbeen og er nu kaster. 

Taber en deltager sin hat på et hvilket som helst tidspunkt skal den øl bundes.

\end{multicols}

\end{document}