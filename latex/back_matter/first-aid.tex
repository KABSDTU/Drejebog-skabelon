\documentclass[../../../main.tex]{subfiles}
\begin{document}

\begin{multicols}{2}
\section{First Aid}
\subsection{First aid box:}
A first aid box is packed with the bathtub and is placed at \placeholder.


\subsection{Generel firstaid}
{\Huge \textcolor{red}{NOT YET TRANSLATED}\\}
Her er en liste over ting I kan komme ud for på rusturene:
\\\\
\textbf{Fremmedlegemer i luftvejene}: Ved delvist blokeret (patienten kan godt sige noget) opfordres der til hoste og evt. slag med en hånd mellem skuldrebladene,samtidig med hoste. Ved total blokering anbefales det, at der veksles mellem at give 5 slag mellem skuldrebladene og 5 hårde stød i bughulen op mod mellem gulvet med \textbf{Heimlichmanøvren}: (armene rundt om patient der bøjer sig forover). Når man har udøvet Heimlichmanøvren på en person, skal denne person ALTID på skadestuen. Uanset antallet af stød.
\\\\
\textbf{Hjernerystelse}: Kommer til udtryk ved en eller flere af følgende symptomer: Hovedpine, Kvalme, Opkast, Svimmelhed, Synsforstyrrelser, Træthed og tab af korttidshukommelse. \\
Førstehjælpen er:\\
Læg personen for at slappe af/sove og hold tæt observation i mindst 24 timer, f.eks. ved at vække personen en gang i timen. Dette er for at sikre, at tilstanden ikke forværres. Bliver personen tiltagende dårlig, skal der søges skadestue eller læge hurtigst muligt.
\\

\textbf{Forbrænding:} Mindre forbrændinger skal straks køles ned med så koldt vand som patienten kan holde ud, indtil det ikke gør ondt længere. Sværere forbrændinger (2.grads kan ses ved at der er vabler/sår og 3. grads vil huden være læderagtig og blødende) skylles med tempereret vand (ca. 18-19 grader), der alarmeres og skylning fortsætter under transport til videre behandling.
\\

\textbf{Hedeslag/solstik:} Symptomer: hovedpine, svimmelhed og træthed. Huden vil være varm og lyserød. Patienten vil svede og være konfus. Førstehjælpen er at få patienten i skygge og fjern/løsne stramt tøj. Læg koldt omslag ved pande, nakke, håndled, lyske og ankler, og giv dem noget at drikke.
\\

\textbf{Knoglebrud:} Ved ben- eller bækkenbrud tilkald ambulance og støt bruddet i findestillingen. Ved mindre brud kan man køre selv og holde bruddet i ro, evt. med trekant tørklæde.
\\

\textbf{Forstuvning:} Opstår ved for stor og forkert belastning af et led, symptomerne kan afhjælpes ved at benytte sig af \textbf{R.I.C.E}-princippet:
\begin{itemize}
    \item \textbf{R} = Rest. Hold det beskadigede område i ro i det første døgn.

    \item \textbf{I} = Ice. Afkøl med is(køleposer kan bruges) eller koldt vand. Køl i ca. 30 min og hold derefter en times pause i de første seks timer efter skaden. Er den stor kan man fortsætte i op til et døgn.

    \item \textbf{C} = Compression. Sørg for at lægge pres på det beskadigede sted øjeblikkeligt, hold det i ca. 10-20 min og anlæg derefter et støttebind.

    \item \textbf{E} = Elevation. Løft det beskadigede område over hjerte højde.
\end{itemize}

\textbf{Snitsår:} Rens med renseservietter og sæt plaster på, brug evt. lille kompresforbinding. Løft for at standse blødning.
\\

\textbf{Næseblod:} Pres 2 fingre på næsen og bøj hovedet let forover. Læg evt en kold klud eller is over næsen, for at få blodkar til at trække sig sammen. Hvis blødningen ikke er stoppet efter en halv time så søg læge eller skadestue.
\\

\textbf{Tæge:} Fjern tægen med tægetang. Man skal aldrig mase eller nulre tægen. Kan den ikke fjernes, søg læge.
\\

\textbf{Astma:} Ved anfald, støt personen i at sidde eller stå. De skal ikke ligge ned, da dette kan forværre åndenøden. Armene over hovedet kan også hjælpe. Sørg for frisk luft og giv psykisk førstehjælp. Evt. hjælp med personens egen medicin eller astmaspray.
\\
\textbf{Diabetes:} Skyldes nedsat eller ingen produktion af hormonet insulin, der transporterer sukker fra blod ind i cellerne. Anfald kan skyldes lavt eller højt blodsukker. Ved lavt blodsukker gives noget sødt at spise eller drikke. Ellers trinvis førstehjælp. Er man i tvivl om det er højt eller lavt blodsukker så giv noget sødt alligevel, da det ikke gør den store forskel, hvis blodsukkeret allerede er for højt. Man må ikke selv give insulin til sukkersygepatienter. Hvis de er ved bevidsthed kan man hjælpe, da de som regel selv har styr på det. Ved tvivl spørg.
\\
\textbf{Epilepsi:} Få personen til at ligge ned og løsne stramt tøj om halsen. Beskyt hovedet mod stød. Når krampen ophører lægges personen i stabilt sideleje for at skabe frie luftveje. De vil oftest være meget trætte. Forsøg ikke at stikke noget mellem tænderne, men forhold dig i ro.

\clearpage

\subsection{Førstehjælp fra CPH care}

\subsection*{Fremmedlegeme og slag i øjet} Mange får fremmedlegeme i øjet, f.eks. træsplinter fra en rundsav eller lignende. Dette sker typisk, fordi man glemmer at benytte beskyttelsesbriller. Et fremmedlegeme kan beskadige øjetsm hornhinde med smerte og eventuel infektion til følge.
\\

\textbf{Almindelig førstehjælp:}
Skyl øjet med vand fra næserod og udefter. Hvis fremmedlegemet ikke kan skylles ud, forsøg da forsigtigt at fjerne synlige fremmedlegemer med en vatpind eller med snippen af et stykke klæde. Blænd det ene øje af ved dobbeltsyn.
\\

\textbf{Søg Læge:}
Ved fastsiddende fremmedlegemer i øjet, ved smerter, blødning, dobbeltsyn eller hvis øjet konstant løber i vand.
\\

\subsection*{Fremmedlegeme i huden} Have-, metal-, glas- og træarbejde er ofte årsag til splinter i huden. Hænderne er særligt udsatte. Hvis ikke splinten fjernes kan der opstå infektion. Fremmedlegemer kan også forekomme efter bid/stik fra insekter og fald.
\\

\textbf{Almindelig førstehjælp:} Fjern fremmedlegemet med en pincet eller nål.
Hold området i sæbevand i 15 minutter. Start aldrig med sæbevand ved træsplinter. Splinten vil gå i opløsning og herved blive svær at
fjerne.
\\

\textbf{Fastsiddende fremmedlegeme:} Med en ring lavet af tørklæde eller lignende (donut) støttes fremmedlegemet i såret. Anlæg en løs forbinding ovenpå for at holde fremmedlegemet i ro. 
\\

\textbf{Søg Læge:} Ved fastsiddende fremmedlegemer, f.eks. fiskekroge, forsøg ikke selv at fjerne fremmedlegemet. 
\\
\subsection*{Bid} Bid kan være forårsaget af dyr eller mennesker. Bid kan give muskelskader og der kan opstå infektion i bidsåret. Derfor er det vigtigt at tænke på stivkrampevaccine hvis man bliver bidt. 
\\

\textbf{Almindelig førstehjælp:} Stands blødningen. Vask med vand og sæbe indtil bidet er rent. Anlæg plaster eller forbinding.
\\

\textbf{Søg Læge:} Når huden er gennembrudt, hvis biddet er i ansigtet
eller på halsen. Søg altid læge ved rødme, ømhed, varme eller hævelse. 
\\

\subsection*{Næseblod}Næseblod forekommer oftest ved direkte slag på næsen, eller ved påvirkning fra fremmedlegemer i næsen, som f.eks. fingre.
\\

\textbf{Almindelig førstehjælp:} Den tilskadekomne anbringes siddende med
hovedet let fremover. Bøj hagen ned mod brystet så blodet løber ud og ikke ned i halsen. Klem med to fingre om overgangen mellem den hårde og den bløde del af næsen indtil blødningen standser. Læg en kold klud eller is/ispose over næse og pande. Giv evt. en isterning i munden
\\

\textbf{Søg Læge:} Stopper blødningen ikke inden for en halv time, eller sidder næsen i fejlstilling søg læge eller skadestue.
\\
\subsection*{Skovflåt} Skovflåten er en blodsugende mide, der bider sig fast, så kun bagkroppen stikker ud. Skovflåten findes i buskads, skovbund, lav
bevoksning og lignende. Ved ophold i disse områder vil skovflåten sætte sig
på dyr og mennesker. Efter skovture bør man altid kontrollere for skovflåter.
Skovflåten kan være smittebærer af Borreliabakterien, ca. 50 procent af alle skovflåter indeholder bakterien. Denne bakterie kan i sjældne tilfælde give
hjernebetændelse, hvis infektionen ikke bliver behandlet medicinsk.
\\

\textbf{Almindelig førstehjælp:} Fjern skovflåten med en kniv eller flåt-spartel, uden at trykke på skovflåtens bagkrop. Vask med vand og sæbe. Fjernes skovflåten straks, er der minimal fare for at få Borrelia-infektion. Efter at skovflåten er fjernet, slå en ring rundt om bid stedet, hvis rødme eller udslæt vokser ud over ringen er dette en indikation på at der skal søges læge.
\\

\textbf{Søg Læge:} Ved rødme, ømhed, varme, hævelse eller udslæt omkring det sted hvor biddet forekom.
\\

\subsection*{Insektstik - bi/hveps} Det hyppigste er bi eller hvepsestik, som almindeligvis ikke er farligt, men kan være det, hvis personen er allergisk over for bi-/hvepsestik.
\\

\textbf{Tilkald hjælp:} Ring 1-1-2 ved stik i mund eller svælg, voldsom hævelse, udslæt, utilpashed, åndedrætsbesvær eller hvis personen har kendt allergi med kraftig reaktion ved insektstik.
\\

\textbf{Almindelig førstehjælp:} Er almen tilstanden påvirket, læg da personen ned, pak evt. personen ind i tæpper eller lign. Såfremt personen begynder at fryse. Er almentilstanden ikke påvirket: Fjern brodden med en nål eller ved at skrabe med en kniv (brug ikke pincet) Fjern stramtsiddende smykker og tøj. Nedkøl med ispose i 30 min. Ved stik i mund eller svælg, køl med is.
\\

\textbf{Søg Læge:} Ved en kraftig reaktion, hævelse, udslæt, utilpashed, åndedrætsbesvær, eller ved kendt allergi. 
\\

\subsection*{Stik og snitsår} Mindre stik- og snitsår, forårsaget af skarpe genstande, splintret glas, skarpe metalplader osv. Ses ofte i forbindelse med madlavning.
\\

\textbf{Almindelig Førstehjælp:} Klem sårrandene fast sammen og løft det blødende sted over hjerteniveau indtil blødningen er standset. Vask med vand og sæbe. Såret samles med strips eller plaster.
\\

\textbf{Mindre sår og hudafskrabninger:} Fald på asfalt, beton eller i skoven forårsager ofte hudafskrabninger. Udækket hud er særligt udsat.
\\

\textbf{Almindelig førstehjælp:} Vask med vand og sæbe. Brug en blød svamp eller børste ved rengøring. Lad såret lufttørre. Anlæg et plaster eller forbinding hvis det bløder eller væsker fra såret. Et sår der ikke bløder eller væsker heler bedst utildækket.
\\

\textbf{Søg læge:} Hvis såret er dybere end 1/2 cm og mere end 2 cm langt, hvis såret gaber eller blødningen ikke standser efter 30 min. Søg læge ved rødme, ømhed, varme eller hævelse. 
\\

\subsection*{Tandskader} I forbindelse med læsioner i ansigtet forekommer der
ofte skader på tænderne. Udslåede tænder kan i mange tilfælde reddes og skal derfor altid medbringes urensede til tandlæge eller skadestue. Der vil i forbindelse med tandskader ofte forekomme stærke smerter i munden.
\\

\textbf{Almindelig førstehjælp:} Er tanden løs, lad den sidde. Slås et stykke af en tand eller slås en tand helt ud, opbevares tanden i saltvand (en teske til en liter vand) eller mælk. Det er vigtigt at udslåede tænder opbevares fugtigt. I nødstilfælde, ved mangel på saltvand eller mælk, kan en udslået tand, på vej til tandlægen, opbevares i munden, mellem underlæben og tænderne i undermunden, så den ikke sluges. Hos børn og bevidsthedspåvirkede bør man ikke placere tandstumperne i munden af frygt for at personen sluger dem.
\\

\textbf{Søg læge:} Søg tandlæge eller skadestue ved tandskader. Ved udslåede tænder anbefales det at medbringe den udslåede tand. 
\\
\subsection*{Bjørneklo (giftig plante)} Bjørneklo er en stor giftig vildtvoksende plante, som forekommer hyppigt i Danmark.
\\

\textbf{Symptomer:} Ætsende sår med blæredannelse, svie og smerte. Der kan i særligt voldsomme tilfælde gå op til seks måneder, før sårene heles.
\\

\textbf{Almindelig førstehjælp:} Skyl med vand. Fjern stramtsiddende smykker og
tøj. Vask med vand og sæbe. Anlæg plaster eller
forbinding. Hudområder der har været i berøring med bjørneklo må ikke udsættes for sollys. Sollys forstærker virkningen af plantens gift.
\\

\textbf{Ved bjørneklosaft i øjet:} Åbn øjet og skyl med vand fra næserod og udad. Hvis øjenskyllevæske er til rådighed, anvendes dette.
\\

\textbf{Søg læge:} Ved voldsom hævelse, ætsende sår med blæredannelser, utilpashed eller vejrtrækningsbesvær. Søg altid læge ved bjørneklosaft i øjet. 
\\

\subsection*{Astma} Astma er en sygdom i lungerne, der typisk er udløst af en allergisk reaktion, eller hårdt fysisk arbejde og som kan give anfaldsvis besværet vejrtrækning, hvor der også ses øget slimproduktion og muskelsammentrækninger i bronkier. Kommer vedkommende ikke under behandling, kan
situationen forværres. Behandlingen er i dag så god, at langt de fleste kan undgå/begrænse voldsomme anfald.
\\

\textbf{Symptomer:} Hvæsende og besværet udånding, nedsat udåndingskapacitet, angstpræget adfærd/uro, talebesvær p.g.a. vejrtrækningsproblemer og eventuel blåfarvning af læber, ører og næsefløje (iltmangel).
\\

\textbf{Livreddende førstehjælp:} HLR ved livløshed.
\\

\textbf{Tilkald hjælp:} Ring 1-1-2
\\

\textbf{Almindelig førstehjælp:} Hjælp personen med evt. at tage sin astmamedicin. Hold personen i ro, anbring ham siddende, stående eller i aflastende rygleje, afhængigt af ønske. Løsn stramtsiddende tøj, sørg for frisk luft, fjern røg og stærke lugte. Bevidstløse med vejrtrækning anbringes i stabilt sideleje.
\\
\subsection*{Blodigt opkast}Blødning i mavesækken er den hyppigste årsag til blodigt opkast. Årsagen kan være sygdom, som f.eks. mavesår eller blødning fra spiserøret. Der kan enten være tale om en frisk lys blødning eller kaffegrumsfarvet opkast, alt efter hvor blødningen stammer fra. Blødningen kan være så kraftig, at personen udvikler shock.
\\

\textbf{Symptomer:} Lysfarvet blodigt eller kaffegrumsfarvet opkast. Smerter fra maven. Alment påvirket med hurtig og svag puls, bleg og koldsvedende hud (shock).
\\

\textbf{Livreddende Førstehjælp:} HLR ved livløshed.
\\

\textbf{Tilkald hjælp:} Ring 1-1-2
\\

\textbf{Almindelig førstehjælp:} Anbring den syge i sideleje. Den syge må ikke få noget at spise eller drikke.
\\

\subsection*{Febersygdomme} En række sygdomme viser sig ved, at den syge får
feber. Feber er som regel tegn på infektion, og ofte vil man umiddelbart af symptomerne kunne slutte sig til infektionens art. Uden termometer kan man let skaffe sig et indtryk af, om en person har høj feber ved at førstehjælperen anvender sin egen hånd som temperaturføler.
\\

\textbf{Symptomer:} Varm/brændende hud, klam svedende hud, kulderystelser ved voldsom temperaturstigning og træthed.
\\

\textbf{Almindelig førstehjælp:} Rigeligt at drikke (kolde eller varme væsker efter behov) Ved kulderystelser tildækkes den syge med ekstra tæppe, indtil ophør af rystelser, herefter afdækkes vedkommende. Ved høj temperatur afdækkes den syge ved, at fjerne dynen og kun tildække med et lagen. Ved febersygdomme i mere end 3 dage, bør læge kontaktes.
\\

\subsection*{Maveinfektioner} Andre sygdomme, der kan give pludselig feber, er
maveinfektioner. Maveinfektioner opstår ofte som følge af madforgiftning, der typisk stammer fra dårligt drikkevand fordærvet mad, eller dårlig håndhygiejne.
\\

\textbf{Symptomer:}Mavesmerter, som mavekneb, smerter forskellige steder i maven, opkast og diarré.
\\

\textbf{Almindelig førstehjælp:} Rigeligt at drikke i form af ikke kolde væsker. Eksempelvis cola eller „turistdrik“ (1 liter vand tilsat 2 spiseskeer sukker og 1/2 teske salt). Efter hver gang diarré drikkes 1 glas væske. Når opkast er ophørt, kan der indtages reven æble, banan, kiks eller lignende.

\textbf{Forebyggelse:} Drik kun renset vand. Vask spisebestik og service
grundigt. Gennemsteg/kog maden. Hav en god håndhygiejne, især omkring toilet besøg og buffeter.
\\

\textbf{Livreddende førstehjælp:} HLR ved livløshed.
\\

\textbf{Tilkald hjælp:} Ring 1-1-2
\\

\subsection*{Sukkersyge (diabetes)} Sukkersyge er en tilstand primært karakteriseret ved enten en for lille produktion af insulin, en nedsat følsomhed for den producerede insulin eller nedsat tolerance for glucose. Herved vil kroppen ikke være i stand til under alle omstændigheder at holde blodsukkeret på et for kroppen gunstigt niveau.
\\

\textbf{}Der skelnes mellem 2 typer af sukkersyge. 
\\

\textbf{}Symptomerne på for højt og for lavt blodsukker kan ligne hinanden meget. Hvis man er i tvivl om blodsukkeret er for højt eller for lavt, behandles den syge med sukker, idet et for lavt blodsukker hurtigt kan udvikle sig til en livstruende tilstand. Ved tvivl spørg.
\\

\subsection*{For højt blodsukker} Skyldes ophør med insulinbehandling, infektioner/graviditet eller alvorligt diætbrud.
\\

\textbf{Symptomer:} Udvikles langsomt over timer til dage: Tørst, mathed, kvalme, opkastninger, mavesmerter, omtågethed, bevidstløshed.
\\

\textbf{Tilkald hjælp:} Ring 1-1-2
\\

\textbf{Almindelig førstehjælp:} Er personen vågen gives evt. hjælp til at tage insulin\\

\subsection*{For lavt blodsukker} Skyldes for lidt mad, for meget insulin, hård fysisk aktivitet eller større mængder alkohol.
\\

\textbf{Symptomer:} Udvikles over minutter til timer: Sult, mathed, svimmelhed, bleg, koldsved, rysten, omtågethed, sløret tale, mærkelig evt. aggressiv adfærd, kramper, bevidstløshed.
\\

\textbf{Tilkald hjælp:} Ring 1-1-2
\\

\textbf{Almindelig førstehjælp:} Er personen vågen gives sødt at drikke eller spise f.eks. sød saft, juice eller sukker
\\

\subsection*{Epilepsi (kramper)} Epilepsi er en alvorlig sygdom i centralnervesystemet. Epilepsi viser sig ved pludselige anfaldsvise kramper, der er ledsaget af kortere eller længere bevidstløshed. Anfaldets varighed er normalt 2 til 10 minutter eller længere, før bevidstheden vender tilbage. Efter anfaldet vil personen være meget træt og øm i kroppen, ligesom der kan være hovedpine og trang til søvn.
\\

\textbf{Symptomer:} Rykvise kramper, bevidstløshed, evt. vandladning, fråde om munden og blåfarvning af læber, ører og næsefløje (iltmangel).
\\

\textbf{Stands ulykken:} Læg den syge ned, fjern alt omkring personen og beskyt hovedet mod stød så længe der er krampe, med evt. en jakke.
\\

\textbf{Livreddende Førstehjælp:} HLR ved livløshed.
\\

\textbf{Tilkald hjælp:} Ring 1-1-2
\\

\textbf{}
Til personer, der er kendt som epilepsipatienter
tilkaldes hjælp (1-1-2), hvis anfaldet varer mere end
5 minutter eller der optræder flere anfald umiddelbart
efter hinanden, uden at den syge kommer til
bevidsthed.
\\

\textbf{Almindelig førstehjælp:} Løsn stramtsiddende tøj og anbring den syge i stabilt sideleje når anfaldet er ophørt. Forsøg ikke at stikke noget mellem tænderne på den syge og forsøg ikke at give noget at drikke.
\\
\subsection*{Hjernehindebetændelse (Meningitis)}
Meningitis er en betændelses tilstand der sætter sig i hjernehinden og udvikler sig utroligt hurtigt, inden for blot få timer. Ca. 10 procent af befolkningen er bærere af bakterien, der sider i næsen og svælget og smitter igennem nys og hoste. Det er ikke alle der smittes der bliver syge, grundet en del af befolkningens immunforsvar kan nedkæmpe bakterien, men hvis en ikke immun person smittes af sygdommen, er overlevelses chancen, hvis ubehandlet, meget lille og selv ved behandling er der stadigvæk en dødelighed på mellem 5-10 procent. Det er den bakterielle meningitis der er farligst og kræver omgående behandling med antibiotika direkte i blodårerne.
\\

\textbf{Symptomer:} Meningitis symptomer viser sig ved nakkestivhed, hvor det gør ondet hvis man prøver at bøje nakken. Mange får voldsom hovedpine, kvalme og opkastning. Meningitis patienter får ofte meget høj feber (39 – 41 grader). Der kan ses småblødninger i huden på størrelse med knappenålshoveder, dette er et meget alvorligt tegn, der tyder på at sygdommen har bredt sig til blodet og at patienten skal behandles omgående. Disse blødninger kan skelnes fra andet udslæt, ved at de ikke forsvinder når man trykker på dem med en finger.
\\

\textbf{Livreddende førstehjælp:} HLR ved livløshed – undgå mund til mund.
\\

\textbf{Tilkald hjælp:} Ring 1-1-2, bring patienten hurtigst muligt på
hospitalet.
\\

\textbf{Omgivelserne:} De mennesker der har været omkring patienten, ifm.
udbruddet, vil blive sat i karantæne, undersøgt for meningitis og tilbudt en vaccine.
\\
\subsection*{Forbrændinger} Ved en kraftig varmepåvirkning af huden
forekommer forbrændinger, Forbrændinger inddeles i 3 grader. Ved forbrændinger i ansigtet, er der risiko for indvendige forbrændinger af luftvejene.
\\

\textbf{1. Grads forbrænding:} Medfører rødme af huden. Den kenders typisk i form af en solskoldning. 1. grads forbrændinger er ufarlige men giver sviende smerte og let hudafskalning efter et par dage.
\\

\textbf{2. Grads forbrænding:} Medfører stærke smerter. Efter en halv til en hel time efter varmepåvirkningen dannes der væskefyldte vabler, ved dybere skader kan der være blod i vablerne. 2. grads forbrændinger kan hele op uden men, det er vigtigt at infektion i det forbrændte område undgås.
\\

\textbf{3. Grads forbrænding:} Medfører forkulning, der er ingen smerter, da nerverne er brændt væk. Ved skoldninger ses hvid hud. Der kan være smerter rundt om 3. grads forbrændingen hvor 1. og 2. grads forbrændinger ofte vil fremkomme. En støre forbrænding, en hel arm eller mere, kan forsage en shock tilstand på grund af væskeudsivning. 
\\

\textbf{Stands ulykken:} Læg den til skade kommende ned på jorden og kvæl
ilden med vand, eller et tæppe. Sluk fra hovedet og ned efter. Alternativt nødflyt personen fra det brændende sted.
\\

\textbf{Livreddende Førstehjælp:} HLR ved livløshed.
\\

\textbf{Tilkald hjælp:} Ring 1-1-2
\\

\textbf{Almindelig førstehjælp:} Nedkøl det forbrændte område med tempereret vand (stue temperatur) indtil smerten er væk. En våd klud kan lægges på og vendes hyppigt, hvis der er begrænset tilgang til vand. Giv den til skade kommende noget at drikke, undgå alkohol, tobaksvarer og kaffe, da dette kan virke vandrivende og kar udvidende.
\\

\textbf{Søg læge:} Ved anden og tredje grads forbrændinger eller ved forbrænding omkring luftvejene, da der vil være risiko for indvendige brandskader. 
\\
\subsection*{Ætsninger} Nogle stoffer, f.eks. opvaske midler til opvaskemaskiner, el. syre fra bilbatterier, kan give ætsninger, der kan resultere i stærke smerter, beskadiget hud, der er enten bleg eller brunlig og i særligt slemme tilfælde kan ende i en shock tilstand.
\\

\textbf{Stands ulykken:} Undgå selv at berøre eller indånde ætsemidlet.
\\

\textbf{Livreddende førstehjælp A-B-C:} 

\textbf{}HLR ved livløshed.
\\
\textbf{}Anvend beskyttelsesmaske ved kunstigt åndedræt for ikke at komme i kontakt med det ætsende stof, eller giv kun hjertemassage.
\\

\textbf{Almindelig førstehjælp:} 

\textbf{Ved udvendige ætsninger:} Skyl med rigelige mængder af vand indtil smerten er væk, dog mindst 30 minutter. Fjern tøj som er gennemvædet af det ætsende stof.
\\

\textbf{Ved ætsning af øjet:} Åben øjet og skyl med vand fra næserod og ud. Hvis øjenskyllevæske er til rådighed, anvendes dette.
\\

\textbf{Ved indvendige ætsninger:} Skyl og rens mund grundigt med vand. Giv vand eller mælk at drikke i små slurke. Undgå at fremprovokere opkast. Aflever så vidt muligt emballagen fra det ætsende stof til redningsmandskabet.
\\

\textbf{Søg læge:} Hvis det ætsende stof er kommet i øjet, hvis det ætsende stof er blevet indtaget, eller hvis patienten udviser symptomer på shock (bleg, koldsved, overfladisk vejtrækning, omtåget)
\subsection*{Varmepåvirkning (dehydrering)} En persons krop kan ved pludselige
temperaturforandringer, f.eks. i forbindelse med rejser, årets første hedebølge eller længerevarende ophold i solen uden nok væske, reagere med pludselig udmattelse og besvimelse. Den normale legemstemperatur er 37 C. Ved en legemstemperatur på 39,5 C påvirkes centralnervesystemet.
Hedeslag er en livstruende tilstand hvor kroppen mister evnen til at nedkøle sig selv, dette kan også opstå ved manglende væskeindtag/væskemangel.
\\

\textbf{Symptomer:}
Kramper i arme og ben, svedig og kold hud, hovedpine, opkast, sløv og forvirret, ophørende sved produktion, hallucinationer og bevidstløshed.
\\

\textbf{Stands ulykken:} Beskyt mod sol og varme, anbring patienten i
skyggen eller et køligt sted, løsn stramtsiddende tøj og aftag varmende beklædning, sko, kasket m.m.
\\

\textbf{Almindelig førstehjælp:} Giv rigeligt at drikke i små slurke, gerne sødt. Undgå kolde drikke, da dette kan give mavekramper og opkast. Fugt personens hud med vand og vift luft henover. Anbring den varmepåvirkede med ben og arme i koldt vand, eller læg personen ned og vikle våde tørklæder om ankler, håndled, isse og nakke. Undgå nedkøling med isposer.
\\

\textbf{Søg læge:} Ved bevidstløshed, hvis patientens temperatur bliver
ved med at stige eller hvis patienten ikke får det bedre efter at være kommet i skygge og fået sødt at drikke. 
\\
\subsection*{Kuldepåvirkning} Årsagerne til underafkøling er alt fra kulde, væde,
blæst, mangel på bevægelse og alkoholpåvirkning. Hvis en person opholder sig i vand, vil underafkølingen opstå meget hurtigere.
\\

\textbf{Symptomer:} Symptomer på underafkøling er kulderystelser, bleg
og evt. blåmarmoreret hud, nedsat bevægelighed i ben og arme og især hænder og føder. Ved svær underafkøling vil der opstå dyb bevidstløshed og risiko for hjertestop.
\\

\textbf{Stands ulykken:} Beskyt mod kulde og væde, anbring den kuldepåvirkede i læ, hvis personen er omtåget eller bevidstløs skal denne transporteres i vandret stilling, da der ellers er risiko for hjertestop.
\\

\textbf{Almindelig førstehjælp:} Opvarm den kuldepåvirkede,
\\

\textbf{hvis personen er vågen:} Aftag vådt tøj. Indpak personen i redningstæppe med sølvsiden indad og herefter i tæpper eller andet. Giv varmt sødt at drikke (ikke kaffe og alkohol, da dette er karudvidende).
\\

\textbf{Fysisk aktivitet:} Aktiver personen fysisk f.eks. med kuskeslag eller march.
\\

\textbf{Kammeratopvarmning:} Lig f.eks. tæt sammen i en sovepose.
\\

\textbf{Søg læge:} Ved bevidstløshed, hvis personen er omtåget eller hvis personen ikke længere har kulderystelser. Bevidstløse skal altid opvarmes af en læge. 
\\
\subsection*{Hovedskader (hjernerystelse)} Hoved og ansigtslæsioner opstår ofte i forbindelse med vold, trafikulykker, fald og sportsulykker.
\\

\textbf{Symptomer:} Symptomer på en hjernerystelse er Smerte, kortvarig
bevidstløshed, hukommelsestab lige efter ulykken, kvalme og evt. opkast. Ved kraniebrud, kan der ses blødning eller udsivning af klar væske fra øre og/eller næse.
\\

\textbf{Stands ulykken:} Tal til den til skade kommende forfra så denne ikke
drejer hovedet. Støt hoved og nakke i finde stillingen.
\\

\textbf{Almindelig førstehjælp:} Ved mistanke om brud på nakke eller ryg, sikres den tilskadekommende i findestillingen. Ved åbent kraniebrud anlægges en løs forbinding for at standse blødningen. Ved mistanke om hjernerystelse overvåges den tilskadekommende, en gang i timen skal den tilskadekommende redegøre for navn, dag og sted.
\\

\textbf{Faresignaler ved hjernerystelse:} Hvis personen har været bevidstløs i mere end få minutter. Hvis den tilskadekommende efter opvågning, igen
bliver søvnig eller ukontrollerbar. Hvis den tilskadekommende efter bedring igen får det bedre.
\\

\textbf{Søg læge:} Hvis patienten ikke kan redegøre for hvor patienten
er, eller hvis patienten udviser et eller flere faresignaler. 
\\
\subsection*{Knoglebrud} Årsagerne til et brud kan være direkte slag, fald, vrid, skud eller overbelastning. Der skelnes mellem 2 typer af knoglebrud. Lukket knoglebrud, hvor huden er hel og ikke truet af gennembrud fra knogleenderne. Åbent knoglebrud, hvor huden er gennembrudt af knogleenderne, eller trues af gennembrud.
\\

\textbf{Symptomer:} Smerte, hævelse, misfarvning, fejlstilling, nedsat
bevægelighed, føleforstyrrelser, løshed, blødning og koldsvende hud (shock).
\\

\textbf{Stands ulykken:} Undgå unødvendig bevægelse af det skadede
område. Støt det skadede område med hænderne, eller understøt det med beklædning, tæpper etc. I findestillingen.
\\

\textbf{Almindelig førstehjælp:} Åbne knoglebrud forbindes som sår med
fremmedlegeme. Med en ring lavet af tørklæde eller lignende (dougnut) støttes
knoglenderne i såret. Anlæg løs forbinding. Alle knoglebrud immobiliseres enten mod underlaget, kroppen eller i en skinne. Ved bevidstløshed med vejrtrækning anbringes den tilskadekomne i aflåst sideleje i forhold til skaden
\\

\textbf{Søg læge:} Søg altid læge ved brækkede lemmer. Ved komplicerede brud, f.eks. på ben, ryg eller Nakke skal der ringes 1-1-2 da patienten skal afhentes af en ambulance. 
\\
\subsection*{Forstuvninger og muskelskader} Forstuvning er en påvirkning af leddet, hvor det bliver overstrakt for herefter at vende tilbage til normal stilling. Disse skader kan opstå ved idræt, fald eller stød. Påvirkningen af leddet forårsager blødning og væskeudtrækning i leddet og det omgivende væv med deraf følgende smerte og hævelse. Muskelskader kan opstå ved slag mod et kropsområde eller ved overrivning af muskelceller (fibersprængning). Herved kan der opstå en kraftig blødning i muskulaturen.
\\

\textbf{Symptomer:} Smerte/ømhed, hævelse, misfarvning og nedsat
bevægelighed.
\\

\textbf{Almindelig førstehjælp:} RICE princippet bruges til at behandle forstuvninger, jo hurtigere RICE behandlingen påbegyndes, jo bedre er virkningen. Rest: Stop fysisk aktivitet, for ikke at forværre skadens omfang.
\\

\textbf{Ice – Compression:} Is og kompression tryk direkte på det skadede område for at standse blødningen. Anlæg 4-6 rundture elastikbind stramt omkring det skadede område, men ikke så stramt, at det forhindrer blodcirkulationen. Ispose anlægges på det skadede område, oven på elastikbindet. Isposen fikseres med den resterende del af bindet. Kulde har en dæmpende effekt på blodcirkulationen og herved på blodudsivning og hævelse i vævet. Kulde er også smertelindrende. Kuldebehandling foretages i op til 20 min. 1-4 gange efter behov/smerte.
\\

\textbf{Elevation:} Løft det skadede område over hjerteniveau. Ved mistanke om muskeloverrivning, knoglebrud og ledbåndsskade skal den tilskadekomne straks transporteres til læge eller skadestue.
\\
\end{multicols}
\end{document}