\documentclass[../../../main.tex]{subfiles}
\begin{document}

\begin{multicols}{2}
\section{First Aid}
\subsection{First Aid Kit:}
We have a first aid kit in the bath tub.


\subsection{General First Aid}
The following is a list of injuries that can occur on the rustrips:

\textbf{Foreign objects in the airways}: If partially blocked (the patient can speak) it is advised to induce a cough, possibly to hit the patient between the shoulder blades. In case of a total block, you should administer 5 knocks between the shoulder blades and 5 knocks to the abdominal cavity, using the \textbf{Heimlich maneuver}: (arms around the patient, who is bent forward). After utilizing the Heimlich, the patient must be taken to the emergency room. Regardless of the number of times the number of repeats.
\\\\
\textbf{Concussion}: Manifests with one or more of the following symptoms: Headache, nausea, vomiting, light-headedness, reduced vision, fatigue and loss of memory. \\
The treatment is as follows:\\
Lay the patient down to rest/sleep under continued supervision, for at least 24 hours. If possible, wake the patient once an hour. This is to make sure the condition doesn't worsen. If the patient feels worse over time, they should be taken to the emergency room.    
\\

\textbf{Burns:} Lesser burns should be cooled immediately, with water as cold as the patient can tolerate, until the pain ceases. More severe burns (2nd degree can be identified with blisters/sores. With 3rd degree burns, the skin turns leathery and bleeds) should be continually rinsed with running water, at room temperature (18-19 celsius). Paramedics are called, and rinsing continues under transport to further treatment.
\\

\textbf{heatstroke:} Symptoms: Headache, light-headedness and fatigue. The skin will be warm and pink. The patient will sweat and be confused. The treatment consists of laying the patient down, in the shade, then removing/loosening tight-fitting clothes. Put a wet cloth on the forehead, back of the neck, wrists, ankles and groin. Then give them something to drink.
\\

\textbf{Bone fractures:} In case of leg or pelvis fracture, call an ambulance, and put support on the break in the position in which you found it. With smaller fractures you can drive the patient to the emergency room yourself, keeping the fracture at rest, if needed; use a "triangle scarf".
\\

\textbf{Sprains:} Happens when too much force is applied on a joint. The symptoms can be relieved using the  \textbf{R.I.C.E}-principle:
\begin{itemize}
    \item \textbf{R} = Rest. Keep the sprained joint at rest for the first 24 hours.

    \item \textbf{I} = Ice. Cool the joint with ice (Frozen bags can be used) or cold water. Cool for about 30 min. Hereafter let it rest for an hour. Repeat this for the first six hours after the injury. In severe cases, repeat for 24 hours.

    \item \textbf{C} = Compression. Apply pressure on the affected joint immediately, keep pressure on for 10-20 min. Then apply a support bandage.

    \item \textbf{E} = Elevation. Lift the affected area above heart height.
\end{itemize}

\textbf{Cuts:} Clean with wetwipes, and apply band-aid. Situationnally use a small compression bandage. Lift the cut above heart height, to lessen bleeding.
\\

\textbf{Nosebleed:} Press two fingers against the nose, and bend the head slightly forwards. If needed, put a cool cloth or ice over the nose, to get the capillaries to contract. If the bleeding continues, for more than 30 min. take the patient to a doctor/emergency room.
\\

\textbf{Tick bites:} Remove the tick, with tick-plyers. Do not squeeze or fiddle the tick. If it can't be removed, seek out a doctor.
\\

\textbf{Astma:} In case of an asthma attack, support the patient when sitting or standing. They should not lay down, this can worsen the condition. Holding their hands above the head can also help. Make sure they can get fresh air, and give psychological first aid. If needed, help the patient administer their asthma spray/medicine.
\\
\textbf{Diabetes:} Due to decreased or no production of the hormone insulin, which transports sugar from blood into the cells. Seizures can be caused by low or high blood sugar. At low blood sugar, something sweet is given to eat or drink. Otherwise step by step first aid. If you are in doubt about whether it is high or low blood sugar, give something sweet anyway, as it does not make much difference if the blood sugar is already too high. Do not give insulin to diabetic patients yourself. If they are conscious, you can help, as they usually have control over it themselves. When in doubt, ask them.
\\
\textbf{Epilepsy:} Have the person lie down and loosen tight clothing around the neck. Protect their head from shocks. When the spasm ceases, the person is placed in a stable lateral position to create free airways. They will most often be very tired. Do not try to stick anything between their teeth, but stay calm.

\clearpage

\subsection{First Aid from CPH Care}

\subsection*{Foreign objects and blows to the eyes} Many get foreign body in the eye, e.g. wood splinters from a circular saw or similar. This typically happens because one forgets to wear goggles. A foreign body can damage the cornea of the eye with pain and possible infection as a result.
\\

\textbf{Basic first aid:}
Rinse the eye with water from the root of the nose and outwards. If the foreign body can not be rinsed out, try to carefully remove visible foreign bodies with a cotton swab or with the snippet of a piece of cloth. Apply single eyepatch in case of double vision.
\\

\textbf{Seek out a doctor:}
In case of foreign bodies stuck in the eye, in case of pain, bleeding, double vision or if the eye is constantly running.
\\

\subsection*{Foreign bodies in the skin} 
Gardening, metal, glass and wood work are often the cause of splinters in the skin. The hands are particularly exposed. If the splinter is not removed, infection can occur. Foreign bodies can also occur after insect bites and falls.
\\

\textbf{Basic first aid:} Remove the foreign body with tweezers or a needle. Keep the area in soapy water for 15 minutes. Never start with soapy water on wood splinters. The splinter will dissolve and thus become difficult to remove.
\\

\textbf{Stuck foreign bodies:} With a ring made of scarf or similar (donut), the foreign body is supported in the wound. Apply a loose bandage on top to keep the foreign body still. 
\\

\textbf{Seek out a doctor:} In case of stuck foreign bodies, e.g. fishing hooks, do not try to remove the foreign body yourself. 
\\
\subsection*{Bites} Bites can be caused by animals or humans. Bites can cause muscle damage and infection can occur in the bite wound. Therefore, it is important to think about tetanus vaccine if one gets bitten.
\\

\textbf{Basic first aid:} Stop the bleeding. Wash with soap and water until the bite is clean. Apply band-aid or bandage.
\\

\textbf{Seek out a doctor:} When the skin is pierced, if the bite is in the face or on the neck. Always seek medical attention in case of redness, soreness, heat or swelling.
\\

\subsection*{Nosebleed}Nosebleeds most often occur by direct blows to the nose, or by exposure to foreign bodies in the nose, such as. fingers.
\\

\textbf{Basic first aid:} The injured person is placed seated with his head slightly forward. Bend your chin down towards your chest so that the blood runs out and not down your throat. Squeeze with two fingers around the transition between the hard and the soft part of the nose until the bleeding stops. Place a cold cloth or ice / ice pack over your nose and forehead. If needed put an ice cube in their mouth
\\

\textbf{Seek out a doctor:} If the bleeding does not stop within half an hour, or if the nose is in the wrong position, seek medical advice or an emergency room.
\\
\subsection*{Ticks} The tick is a blood-sucking mite that bites so that only the hind body protrudes. The forest tick is found in shrubbery, forest floor, lichen
vegetation and the like. When staying in these areas, the tick will settle on animals and humans. After picnics, you should always check for ticks. The tick can be a carrier of the Borrelia bacterium, approx. 50 percent of all forest ticks contain the bacterium. This bacterium can in rare cases cause encephalitis if the infection is not treated medically.
\\

\textbf{Basic first aid:} Remove the tick with a knife or tick-pliers, without pressing on the tick's hind body. Wash with soap and water. If the tick is removed immediately, there is minimal risk of getting Borrelia infection. After the tick is removed, wrap a ring around the bite site, if redness or rash grows beyond the ring, this is an indication that a doctor should be consulted.
\\

\textbf{Seek out a doctor:} In case of redness, tenderness, heat, swelling or rash around the place where the bite occurred.
\\

\subsection*{Insect bites - bee/wasp} The most common are bee or wasp stings, which are usually not dangerous, but can be if the person is allergic to bee/wasp stings.
\\

\textbf{Get help:} Call 1-1-2 for stings in the mouth or throat, severe swelling, rash, malaise, difficulty breathing or if the person has known allergies with a strong reaction to insect stings.
\\

\textbf{Basic first aid:} If the general condition is affected, put the person down, pack if necessary the person into blankets or the like. If the person begins to freeze. If the general condition is not affected: Remove the sting with a needle or by scraping with a knife (do not use tweezers) Remove tight-fitting jewelry and clothing. Cool with ice pack for 30 min. In case of stings in mouth or throat, cool with ice.
\\

\textbf{Seek out a doctor:} In case of a strong reaction, swelling, rash, malaise, difficulty breathing, or in case of known allergy. 
\\

\subsection*{Cuts} Minor stab and cut wounds, caused by sharp objects, shattered glass, sharp metal plates, etc. Often seen in connection with cooking.
\\

\textbf{Basic first aid:} Squeeze the wound edges tightly and lift the bleeding spot above heart level until the bleeding has stopped. Wash with soap and water. The wound is closed with strips or plasters.
\\

\textbf{Smaller cuts and scrapes:} Falls on asphalt, concrete or in the woods often cause skin abrasions. Uncovered skin is particularly exposed.
\\

\textbf{Basic first aid:} Wash with soap and water. Use a soft sponge or brush when cleaning. Allow the wound to air dry. Apply a band-aid or bandage if it bleeds or exudes from the wound. A wound that does not bleed or exude heal best uncovered.
\\

\textbf{Seek out a doctor:} If the wound is deeper than 1/2 cm and more than 2 cm long, if the wound yawns or the bleeding does not stop after 30 min. Seek medical attention for redness, tenderness, heat or swelling. 
\\

\subsection*{Tooth injuries} In connection with lesions of the face damage to the teeth often occurs. Lost teeth can in many cases be saved and must therefore always be brought uncleaned to the dentist or emergency room. In connection with tooth damage, there will often be severe pain in the mouth.
\\

\textbf{Basic first aid:} If the tooth is loose, let it sit. If a piece of a tooth is knocked out or a tooth is knocked out completely, the tooth is stored in salt water (one teaspoon to a liter of water) or milk. It is important that knocked-out teeth are kept moist. In an emergency, in the absence of saline or milk, a knocked-out tooth, on its way to the dentist, can be stored in the mouth, between the lower lip and the teeth in the lower mouth, so that it is not swallowed. In children and those with affected consciousness, the pieces of tooth should not be placed in the mouth for fear of being swallowed by the person..
\\

\textbf{Seek out a doctor:} Seek dental advice or emergency room for dental injuries. With knocked out teeth, it is recommended to bring the knocked out tooth.
\\
\subsection*{Bear's paw (Poisonous plant)} Bear's paw is large poisonous plant, common in Denmark.
\\

\textbf{Symptoms:} Corrosive ulcers with blistering, burning and pain. In particularly severe cases, it can take up to six months for the wounds to heal.
\\

\textbf{Basic first aid:} Rinse with water. Remove tight-fitting jewelry and clothing. Wash with soap and water. Apply plaster or
bandage. Skin areas that have been in contact with bear claw must not be exposed to sunlight. Sunlight enhances the effect of the plant's poison.
\\

\textbf{In case of Bear's paw poison in the eyes:} Open the eye and rinse with water from the root of the nose and outwards. If eyewash is available, use that.
\\

\textbf{Seek out a doctor:} In case of severe swelling, corrosive wounds with blistering, malaise or difficulty breathing. Always seek medical attention for Bear's paw poison in the eye.
\\

\subsection*{Asthma} Asthma is a disease of the lungs, which is typically triggered by an allergic reaction or hard physical work and which can cause occasionally difficult breathing, where increased mucus production and muscle contractions are also seen in the bronchi. If the patient does not come under treatment, can
the situation worsens. The treatment today is so good that the vast majority can avoid/limit violent seizures.
\\

\textbf{Symptoms:} Wheezing and difficulty exhaling, decreased exhalation capacity, anxious behavior / restlessness, difficulty speaking due to breathing problems and possible blue discoloration of the lips, ears and nasal wing (lack of oxygen).
\\

\textbf{Lifesaving first aid:} CPR in case of missing vital signs.
\\

\textbf{Get help:} Call 1-1-2
\\

\textbf{Basic first aid:} Help the person with any to take his asthma medicine. Keep the person still, place him sitting, standing or in a relaxing supine position, depending on desire. Loosen tight-fitting clothing, provide fresh air, remove smoke and strong odors. The unconscious with breathing is placed in a stable side position.
\\
\subsection*{Bloody vomiting}Bleeding in the stomach is the most common cause of bloody vomiting. The cause may be disease, such as stomach ulcer or bleeding from the esophagus. It can be either a light red bleeding or coffee grounds coloured vomiting, depending on where the bleeding originates. The bleeding may be so severe that the person develops shock.
\\

\textbf{Symptoms:} Light colored bloody or coffee grounds colored vomit. Pain from the abdomen. Generally affected with fast and weak pulse, pale and cold sweaty skin (shock).
\\

\textbf{Lifesaving first aid:} CPR in case of missing vital signs.
\\

\textbf{Get help:} Call 1-1-2
\\

\textbf{Basic first aid:} Place the patient sideways. The patient must not be given anything to eat or drink.
\\

\subsection*{Fevers} A number of diseases are manifested by the patient getting fever. Fever is usually a sign of infection, and often you will immediately be able to derive the nature of the infection from the symptoms. Without a thermometer, one can easily get an impression of whether a person has a high fever by the first aider using his own hand as a temperature sensor.
\\

\textbf{Symptoms:} Hot / burning skin, damp sweating skin, chills due to sharp temperature rise and fatigue.
\\

\textbf{Basic First aid:} Plenty to drink (cold or hot liquids as needed) In case of chills, cover the patient with an extra blanket until the shaking ceases, then cover him. At high temperatures, the patient is covered by removing the duvet and only covering with a sheet. In case of feverish illness for more than 3 days, a doctor should be consulted.
\\

\subsection*{Stomach infections} Other diseases that can cause a sudden fever are stomach infections. Stomach infections often occur as a result of food poisoning, which typically results from poor drinking water spoiled food, or poor hand hygiene.
\\

\textbf{Symptoms:} Abdominal pain, such as abdominal cramps, pain in various places in the abdomen, vomiting and diarrhea.
\\

\textbf{Basic First aid:} Plenty to drink in the form of non-cold liquids. For example, cola or "tourist drink" (1 liter of water with 2 tablespoons of sugar and 1/2 teaspoon of salt). After each diarrhea, drink 1 glass of fluid. When vomiting has stopped, grated apple, banana, biscuits or the like can be eaten.

\textbf{Prevention:} Drink only purified water. Wash cutlery and dishes thoroughly. Fry / cook the food. Have good hand hygiene, especially around toilet visits and buffets.
\\

\textbf{Lifesaving first aid:} CPR in case of missing vital signs.
\\

\textbf{Get help:} Call 1-1-2
\\

\subsection*{Diabetes} Diabetes is a condition primarily characterized by either an underproduction of insulin, a decreased sensitivity to the insulin produced, or a decreased tolerance to glucose. As a result, the body will not be able to keep blood sugar at a level favorable to the body in any case.
\\

\textbf{}There are two variations of diabetes. Type I is hereditary and is most common amongst younger people. 
\\

\textbf{}The symptoms of too high and too low blood sugar can be very similar. If you are in doubt about whether their blood sugar is too high or too low, treat the patient with sugar, as too low a blood sugar can quickly develop into a life-threatening condition. When in doubt, ask the patient.
\\

\subsection*{Too high blood sugar} Due to discontinuation of insulin therapy, infections/pregnancy or severe dietary interruption.
\\

\textbf{Symptoms:} Develops slowly over hours to days: Thirst, lethargy, nausea, vomiting, abdominal pain, dizziness, unconsciousness.
\\

\textbf{Get help:} Call 1-1-2
\\

\textbf{Basic first aid:} If the person is awake, help them take their insulin.\\

\subsection*{Too low blood sugar} Due to too little food, too much insulin, strenuous physical activity or large amounts of alcohol.
\\

\textbf{Symptoms:} Develops over minutes to hours: Hunger, drowsiness, dizziness, pale, cold sweats, tremors, dizziness, slurred speech, strange possibly aggressive behavior, seizures, unconsciousness.
\\

\textbf{Get help:} Call 1-1-2
\\

\textbf{Basic first aid:} If the person is awake, give them something sweet to drink or eat, e.g. juice or sugar.
\\

\subsection*{Epilepsy (Cramps)} Epilepsy is a serious disease of the central nervous system. Epilepsy manifests itself in sudden seizures accompanied by shorter or longer unconsciousness. The duration of the seizure is usually 2 to 10 minutes or longer before consciousness returns. After the attack, the person will be very tired and sore in the body, just as there may be headaches and urge to sleep.
\\

\textbf{Symptoms:} Jerky cramps, unconsciousness, possibly urination, foaming at the mouth and bluish discoloration of the lips, ears and nostrils (lack of oxygen).
\\

\textbf{Stop the accident:} Lay the patient down, remove everything around the person and protect the head from shock as long as there is cramping, with a jacket for instance.
\\

\textbf{Lifesaving first aid:} CPR in case of missing vital signs.
\\

\textbf{Get help:} Call 1-1-2
\\

\textbf{}
For known epileptics
call for help (1-1-2) if the attack lasts longer than
5 minutes or several seizures occur immediately
one after the other, without the patient coming to
consciousness.
\\

\textbf{Basic first aid :} Loosen tight-fitting clothing and place the patient in a stable side position when the attack has ceased. Do not try to stick anything between the teeth of the patient and do not try to give anything to drink.
\\
\subsection*{Meningitis}
Meningitis is an inflammatory condition that settles in the meninges and develops incredibly quickly, in just a few hours. Ca. 10 percent of the population are carriers of the bacterium that sides the nose and throat and infects through sneezing and coughing. Not everyone who becomes infected becomes ill, due to part of the population's immune system being able to fight off the bacterium, but if a non - immune person is infected with the disease, the chance of survival if untreated is very small and even with treatment there is still a mortality rate between 5-10 percent. It is the bacterial meningitis that is most dangerous and requires immediate treatment with antibiotics directly into the blood vessels.
\\

\textbf{Symptoms:} Meningitis symptoms show up in neck stiffness, where it hurts if one tries to bend the neck. Many people experience severe headaches, nausea and vomiting. Meningitis patients often have a very high fever (39 - 41 degrees). Small bleeds can be seen in the skin the size of pinheads, this is a very serious sign that indicates that the disease has spread to the blood and that the patient should be treated immediately. These haemorrhages can be distinguished from other rashes in that they do not disappear when pressed with a finger.
\\

\textbf{Lifesaving first aid:} CPR in case of missing vital signs.- Avoid mouth contact.
\\

\textbf{Get help:} Call 1-1-2, Bring the patient to a hospital as soon as possible.
\\

\textbf{Surroundings:} Those who have been around the patient at the time of the outbreak, will be put in quarantine, checked for meningitis and offered a vaccine.
\\
\subsection*{Burns}With a strong heat on the skin burns occur, Burns are divided into 3 degrees. In case of burns on the face, there is a risk of internal burns of the airways.
\\

\textbf{1st degree burn:} Causes redness of the skin. It is typically known in the form of a sunburn. 1st degree burns are harmless but cause stinging pain and slight skin peeling after a few days.
\\

\textbf{2nd degree burns:}Causes severe pain. After half to a full hour after the exposure to heat, fluid-filled blisters form, with deeper damage there may be blood in the blisters. 2nd degree burns can heal up without however, it is important that infection in the burned area is avoided.
\\

\textbf{3rd degree burns:}Causes charring, there is no pain as the nerves are burned away. With scalds, white skin is seen. There may be pain around the 3rd degree burn where 1st and 2nd degree burns will often occur. A major burn, an entire arm or more, can cause a state of shock due to fluid leakage. 
\\

\textbf{Stop the accident:}Lay the patient down on the ground and suffocate the fire with water, or a blanket. Turn off from the head and down after. Alternatively, evacuate the person from the burning spot.
\\

\textbf{Lifesaving first aid:} CPR in case of missing vital signs
\\

\textbf{Get help:} Call 1-1-2
\\

\textbf{Basic dirst aid:} Cool the burned area with temperate water (room temperature) until the pain is gone. A wet cloth can be applied and turned frequently if there is limited access to water. Give the injured person something to drink, avoid alcohol, tobacco products and coffee, as this can be diuretic and vasodilating.
\\

\textbf{Seek out a doctor:} In the case of second and third degree burns or in the case of burns around the respiratory tract, as there will be a risk of internal burns.
\\
\subsection*{Corrosion injuries} Some substances, e.g. dishwashing liquid for dishwashers, etc. acid from car batteries, can cause burns that can result in severe pain, damaged skin that is either pale or brownish and in particularly bad cases can end up in a shock condition.
\\

\textbf{Stop the accident:} Avoid touching or inhaling the corrosive substance yourself.
\\

\textbf{Lifesaving first aid A-B-C:} 

\textbf{}CPR in loss of consciousness
\\
\textbf{}Use a protective mask to avoid contact with the corrosive substance or give heart massage only.
\\

\textbf{Basic first aid:} 

\textbf{External corrosive injuries:} Rinse with large amounts of water until the pain is gone, though, at least 30 minutes. Remove clothing soaked in the corrosive substance.
\\

\textbf{Corrosives in the eyes:} Open the eye and rinse with water from the root of the nose and out. If eyewash is available, use this.
\\

\textbf{In case of internal corrosion:} Rinse and clean the mouth thoroughly with water. Give the patient either water or milk in small sips. Avoid provoking vomiting. If possible give the paramedics the label or packing of the corrosive substance, 
\\

\textbf{Seek medical attention:} If the corrosive substance is in the eye if the corrosive substance has been ingested or if the patient shows signs of shock (pale, cold sweat, shortness of breath and cloudiness).

\subsection*{Heat (dehydration):} A persons body can due to sudden temperature changes, for instance in case of traveling, the first heatwave of the year or longer exposure to the sun without water, react with sudden fatigue and fainting. The normal body temperature is 37 C. If the body temperature reaches 39,5 C, it will affect the central nervous system. Heatstroke is a life threatening condition where the body looses the ability to cool itself. This can occur due to dehydration.
\\

\textbf{Symptoms:}
Cramps in arms and legs, sweaty and cold sking, sweat production stopping, hallucinations and unconsciousness. 
\\

\textbf{Stop the accident:} Protect against sun and heat, place the patient in the shade or a cool place, loosen tight clothes and take of hot clothes, shoes, cap etc.
\\

\textbf{Ordinary first aid:} Give the patient plenty to drink in small sips, preferably something sweet. Avoid colds drink, since this can lead to stomach cramps and vomiting. Moisturise the skin of the patient with water and fan air past it. Place the patients arms and legs in cold water, or place the patient on the ground and wrap wet towels around ankles, wrists scalp and neck. Avoid cooling with ice bags.
\\

\textbf{Seek medical attention:} In case of unconsciousness, if the patients temperature keeps on rising or if the patient doesn't start to feel better after getting into the shade and having something sweet to drink. 
\\
\subsection*{Exposure to the cold:} Causes of hypothermia can be general cold, wetness, wind, lack of movement and intoxication. Water will lead to hypothermia way faster, since it will cool a person much faster.
\\

\textbf{Symptoms:} Symptoms of hypothermia are chills, pale and maybe blueish skin, limited movement in legs and arms and especially hands and feet. Servere hypothermia can result in deep unconsciousness and risk of a stroke.
\\

\textbf{Stop the accident:} Protect against the cold and general wetness, place the patient somewhere sheltered from the weather etc. If the patient is clouded or unconscious the patient should be transporten in a horizontal position, else there's a risk of stroke.
\\

\textbf{Ordinary first aid:} Heat the patient slowly.
\\

\textbf{If the person is awake:} Take of wet clothes. Wrap the patient in a rescue-blanket med the silver side inwards and then use blanket or the likes. Give the patient something hot and sweet to drink (not coffee or alcohol, since this will expand your capillaries).
\\

\textbf{Physical activity:} Make the patient do something physical, like walking etc.
\\

\textbf{Buddy-heating:} Lie close in a ex. sleeping bag.
\\

\textbf{Seek medical attention:} In case of unconsciousness, if the patient is clouded or if the person no longer has chills. Unconscious patients should always be heated by a doctor.
\\
\subsection*{Head injury (Concussion)} Head and facial lesions often occur in due to violence, falling and sports accidents.
\\

\textbf{Symptoms:} Symptoms of a concussion are pain, momentary unconsciousness, memory loss just after the accident, nausea and maybe vomiting. In case of a fracture of the skull there will be bleeding or leakage of clear liquid from nose and or ears.
\\

\textbf{Stop the accident:} Talk to the injured coming from in front of them, so they don't turn their head. Brace the head and neck in the position that you found them.
\\

\textbf{Ordinary first aid:} If there's suspicion of a fracture in the neck or back, the patient should be secured in the position they were found. In case of an open fracture on the skull, a loose bandage should be applied to stop the bleeding. In case of suspicion of a concussion the patient should be monitored. Every hour the patient should tell their name, day and location.
\\

\textbf{Warning signs of a concussion:} If the patient has been unconscious for more than a few minutes. If the patient becomes sleepy or uncontrollable after they have woken up. If the patient after feeling better gets worse again. 
\\

\textbf{Seek medical attention:} If the patient can say where they are, or if they show one or multiple warning signs.
\\
\subsection*{Bone fracture} The causes of a fracture can be a direct impact, falling, twisting or an overload. There are two different types of bone fractures open and closed. A closed bone fracture is where the skin is intact and there's no threat of the bones going through the skin. An open bone fracture is where the skin has been pierced by the bone ends, or there's a threat of the bones piercing the skin.
\\

\textbf{Symptoms:} Pain, swelling, discolouration, limited movement, disruption of senses, looseness, bleeding and sweating cold skin (shock).
\\

\textbf{Stop the accident:} Avoid unnecessary movement of the injured area. Brace the injured area with your hands, clothes, blankets, etc. It should be braced in the position that was found.
\\

\textbf{Ordinary first aid:} Open bone fractures should be bandaged as a wound with a foreign object. With a ring made from a scarf or something like that (donut) support the boneends in the wound. Apply a loose bandage. All bone fractures should be immobilized against the ground, body or a brace. If the patient is unconscious but breathing they should be put in a safe position on their side (aflåst sideleje).
\\

\textbf{Seek medical attention:} Always seek medical attention in case of broken limbs. In case of a complicated fracture, ex. on leg, back or neck should result in a 1-1-2 call since the patient must be picked up by an ambulance.
\\
\subsection*{Spraining and muscle injuries}
A sprain is an action where the joint is over stretched and then returning to it's normal position. These kinds of injuries can occur by doing sports, falling or hits. This can cause blood and liquid to go to the joint and the tissue around the joint, which can lead to pain and swelling. Muscle injuries can happen from impacts or tearing of the muscle cells. (Pulled muscle). This can result in heavy internal bleeding in the muscle, which can lead to swelling, bruises and pain in the immediate area.
\\

\textbf{Symptoms:} Pain/soreness, swelling, discolouration and limited movement.
\\

\textbf{Ordinary first aid:} The RICE principal is used the treat sprains, the faster RICE treatment is used the better the effect. Rest: Stop physical activity to stop worsening the injury.
\\

\textbf{Ice – Compression:} Ice and compression. Apply pressure directly to the injury to stop bleeding. Apply 4-6 rounds of elastic tie tightly around the injury without closing blood circulation. An ice bag should be put on top of the elastic tie and then covered by the rest of the elastic tie. The cold has a calming effect on blood circulation and hereby the bleeding and healing in the tissue. The cold is also painkilling. The ice should be applied for 20 minutes followed by 40 minutes of no ice. This should be done 1-4 times depending on the pain.
\\

\textbf{Elevation:} Lift the injured area above the heart. If there's suspicion of muscle tearing, bone fracture or and ligament injury the patient must immediately be transported to a doctor or the ER.
\\

\textbf{Viper bite}
The risk of snake bites in Denmark is generally low, the bites are rarely serious and deaths are very rare if given the right treatment. The mortality rate is stated to be 0.02\% for those who are bitten, which is less than the risk of wasp stings. In the period 1900-2000, only 7 people died as a result of viper bites \\ \\
However, bites should be taken seriously and medical attention should always be sought. Humans respond differently to viper bites. While some experience no symptoms, others will get a severe poisoning and become very ill. The variation is due to, among other things, the amount of poison, age, body weight and the degree of physical activity after the bite. A viper bite is particularly dangerous for young children, especially under 3 years of age, as they, due to their size, are more strongly affected by the poison concentration. In addition, adults with poor health, as well as people who are hypersensitive to snake bites, may be at risk. Likewise, a viper bite can be dangerous for pregnant women, as the fetus may risk dying as a result of the bite. \\ \\
Vipers, with the exception of males in the mating areas in the spring, are not aggressive unless disturbed or insulted. With a defensive bite (scare bite), less poison will usually be injected than with a prey bite. A snake bite therefore does not have to mean viper poisoning. If there are no local symptoms within approx. 1 hour, or the effect becomes only a slight swelling and nothing else, there is no greater risk of poisoning. In the case of real viper poisoning, the area around the bite site begins to swell within minutes. Later, symptoms may occur in the form of sweating, vomiting, dizziness, abdominal pain and diarrhea, as well as due to the influence on the nervous system anxiety, dizziness and possibly unconsciousness. \\ \\
If swelling occurs, the bitten area should be kept as immobile as possible as vigorous activity will disperse the poison. Therefore, keep calm, and keep the bitten body part still, possibly. by laying a bandage that prevents movements e.g. by using a branch or the like as a "rail". It is not advisable to apply varicose veins in the form of bandages or sling over the bite site to reduce the spread of the poison to the rest of the body, as too tight a bandage can cause serious and permanent damage.
If you get a snake bite, do not try to kill or catch the snake, as this can lead to further bites, and if it has already been killed, do not touch it, as snakes can bite for some time after they are apparently dead if affected.


\end{multicols}
\end{document}