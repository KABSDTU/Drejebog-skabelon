\subsection*{Studiefremdriftsreformen}
Ansvarlig: \Mighty
Til oplæg om AUS. 
\subsubsection*{Fag og prøver}
\label{sub:fag_og_pr_ver}
\begin{itemize}
  \item{Tilmelding til fag} Alle studerende \emph{skal} være tilmeldt kurser svarende til 30 \emph{nye(!)} ECTS-point hvert halvår (13-ugers + 3-ugers).
  \item{Tilmelding til prøver} ALLE studerende tilmeldes automatisk eksaminer i de fag, de er tilmeldt. Så snart eftertilmeldingsperioden er slut, kan man ikke vælge fag om, og er i udgangspunktet forpligtet til at bestå eksamen. Hvis man dumper, skal man til reeksamen næste semester eller i en særlig reeksamensperiode.
  \item{Studiestartsprøve} Inden for de to første måneder afholdes en prøve som \emph{skal} bestås for at kunne fortsætte på sit studie.
\end{itemize}

\subsubsection*{SU}
\label{sub:SU}
\begin{itemize}
  \item{Støttetidsregler} Hvis man søger ind på en videregående uddannelse for første gang senest to år efter endt gymnasiel uddannelse, har man 12 SU-klip ud over normeret studietid. Hvis man starter mere end to år efter endt gymnasielt stuie, er man kun berettiget til SU på normeret studietid.
  \item{SU-stop} Hidtil har man kunnet være 12 måneder bagefter med studiet, før SU'en inddrages. Disse regler ændres til 6 måneder, men træder først i kraft pr. \emph{1. september 2016}.
\end{itemize}
