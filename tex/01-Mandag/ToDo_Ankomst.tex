\section{Hyttebumser og KABS ankomst mandag}

\subsection*{Hyttetjek}
\begin{itemize}
\item Undersøg om ALT virker. Især ovne, kogeplader, køleskabe, frysere, toiletter, håndvaske og andre dyre maskiner.
\item Find ud af hvorvidt, der er septiktank eller kloak. Hvis der er septiktank, undersøg da hvordan lokummerne skyller. Hvis ikke vandet flyder perfekt ud, skal der bestilles slamsugning og det skal være med det samme!
\item Undersøg hvert rum for skader, såsom store huller i vægge eller manglende døre osv.
\item Tæl alt op i køkkenet. 
\item Husk foto hvis noget mangler.
\end{itemize}

\subsection{To do i de enkelte lokaler}
\begin{itemize}
\item Ølrum
	\begin{itemize}
	\item Modtage øl
	\item Få det bragt til ølrummet
	\item Sæt kasserne så de står sorteret efter type, det gør det hele meget nemmere senere.
	\item Øl-computer sættes op.
	\end{itemize}
\item Fest/spiserum	
	\begin{itemize}
	\item Musikanlæg sættes op på en måde så folk ikke umiddelbart kan smadre noget
	\item Tjanse-skema ophænges
	\item Ugeplan ophænges
	\end{itemize}
\item Sovesale
	\begin{itemize}
	\item Skilte ophænges på døre til tværholdene
	\item Forbudt-skilte ophænges på vektorernes døre
	\end{itemize}
\item Toiletter
	\begin{itemize}
	\item Der sættes papir på SAMTLIGE toiletter
	\item Sæbe på toiletterne
	\item Skilte med håndvask, drengtoilet og pigetoilet opsættes
	\end{itemize}
\item Generelt
	\begin{itemize}
	\item Resterende skilte hænges op
	\item Husk at tage billeder som dokumentation for hyttens renhed.
	\end{itemize}
\end{itemize}


